



    
\documentclass[11pt]{article}
    
    \usepackage{parskip}
    \setcounter{secnumdepth}{0} %Suppress section numbers
    \usepackage[breakable]{tcolorbox}
    \tcbset{nobeforeafter}
    \usepackage{needspace}
    \usepackage{minted}
    \usemintedstyle{jupyter_python}
    
    \usepackage[T1]{fontenc}
    % Nicer default font (+ math font) than Computer Modern for most use cases
    \usepackage{mathpazo}

    % Basic figure setup, for now with no caption control since it's done
    % automatically by Pandoc (which extracts ![](path) syntax from Markdown).
    \usepackage{graphicx}
    % We will generate all images so they have a width \maxwidth. This means
    % that they will get their normal width if they fit onto the page, but
    % are scaled down if they would overflow the margins.
    \makeatletter
    \def\maxwidth{\ifdim\Gin@nat@width>\linewidth\linewidth
    \else\Gin@nat@width\fi}
    \makeatother
    \let\Oldincludegraphics\includegraphics
    % Set max figure width to be 80% of text width, for now hardcoded.
    \renewcommand{\includegraphics}[1]{\Oldincludegraphics[width=.8\maxwidth]{#1}}
    % Ensure that by default, figures have no caption (until we provide a
    % proper Figure object with a Caption API and a way to capture that
    % in the conversion process - todo).
    \usepackage{caption}
    \DeclareCaptionLabelFormat{nolabel}{}
    \captionsetup{labelformat=nolabel}

    \usepackage{adjustbox} % Used to constrain images to a maximum size 
    \usepackage{xcolor} % Allow colors to be defined
    \usepackage{enumerate} % Needed for markdown enumerations to work
    \usepackage{geometry} % Used to adjust the document margins
    \usepackage{amsmath} % Equations
    \usepackage{amssymb} % Equations
    \usepackage{textcomp} % defines textquotesingle
    % Hack from http://tex.stackexchange.com/a/47451/13684:
    \AtBeginDocument{%
        \def\PYZsq{\textquotesingle}% Upright quotes in Pygmentized code
    }
    \usepackage{upquote} % Upright quotes for verbatim code
    \usepackage{eurosym} % defines \euro
    \usepackage[mathletters]{ucs} % Extended unicode (utf-8) support
    \usepackage[utf8x]{inputenc} % Allow utf-8 characters in the tex document
    \usepackage{fancyvrb} % verbatim replacement that allows latex
    \usepackage{grffile} % extends the file name processing of package graphics 
                         % to support a larger range 
    % The hyperref package gives us a pdf with properly built
    % internal navigation ('pdf bookmarks' for the table of contents,
    % internal cross-reference links, web links for URLs, etc.)
    \usepackage{hyperref}
    \usepackage{longtable} % longtable support required by pandoc >1.10
    \usepackage{booktabs}  % table support for pandoc > 1.12.2
    \usepackage[inline]{enumitem} % IRkernel/repr support (it uses the enumerate* environment)
    \usepackage[normalem]{ulem} % ulem is needed to support strikethroughs (\sout)
                                % normalem makes italics be italics, not underlines
    

    \let\Oldtex\TeX     % provide compatibility with nbconvert <= 5.3.1
    \let\Oldlatex\LaTeX % pre-included in nbconvert > 5.3.1 so redundant
    
    % Colors for the hyperref package
    \definecolor{urlcolor}{rgb}{0,.145,.698}
    \definecolor{linkcolor}{rgb}{.71,0.21,0.01}
    \definecolor{citecolor}{rgb}{.12,.54,.11}

    % ANSI colors
    \definecolor{ansi-black}{HTML}{3E424D}
    \definecolor{ansi-black-intense}{HTML}{282C36}
    \definecolor{ansi-red}{HTML}{E75C58}
    \definecolor{ansi-red-intense}{HTML}{B22B31}
    \definecolor{ansi-green}{HTML}{00A250}
    \definecolor{ansi-green-intense}{HTML}{007427}
    \definecolor{ansi-yellow}{HTML}{DDB62B}
    \definecolor{ansi-yellow-intense}{HTML}{B27D12}
    \definecolor{ansi-blue}{HTML}{208FFB}
    \definecolor{ansi-blue-intense}{HTML}{0065CA}
    \definecolor{ansi-magenta}{HTML}{D160C4}
    \definecolor{ansi-magenta-intense}{HTML}{A03196}
    \definecolor{ansi-cyan}{HTML}{60C6C8}
    \definecolor{ansi-cyan-intense}{HTML}{258F8F}
    \definecolor{ansi-white}{HTML}{C5C1B4}
    \definecolor{ansi-white-intense}{HTML}{A1A6B2}

    % commands and environments needed by pandoc snippets
    % extracted from the output of `pandoc -s`
    \providecommand{\tightlist}{%
      \setlength{\itemsep}{0pt}\setlength{\parskip}{0pt}}
    \DefineVerbatimEnvironment{Highlighting}{Verbatim}{commandchars=\\\{\}}
    % Add ',fontsize=\small' for more characters per line
    \newenvironment{Shaded}{}{}
    \newcommand{\KeywordTok}[1]{\textcolor[rgb]{0.00,0.44,0.13}{\textbf{{#1}}}}
    \newcommand{\DataTypeTok}[1]{\textcolor[rgb]{0.56,0.13,0.00}{{#1}}}
    \newcommand{\DecValTok}[1]{\textcolor[rgb]{0.25,0.63,0.44}{{#1}}}
    \newcommand{\BaseNTok}[1]{\textcolor[rgb]{0.25,0.63,0.44}{{#1}}}
    \newcommand{\FloatTok}[1]{\textcolor[rgb]{0.25,0.63,0.44}{{#1}}}
    \newcommand{\CharTok}[1]{\textcolor[rgb]{0.25,0.44,0.63}{{#1}}}
    \newcommand{\StringTok}[1]{\textcolor[rgb]{0.25,0.44,0.63}{{#1}}}
    \newcommand{\CommentTok}[1]{\textcolor[rgb]{0.38,0.63,0.69}{\textit{{#1}}}}
    \newcommand{\OtherTok}[1]{\textcolor[rgb]{0.00,0.44,0.13}{{#1}}}
    \newcommand{\AlertTok}[1]{\textcolor[rgb]{1.00,0.00,0.00}{\textbf{{#1}}}}
    \newcommand{\FunctionTok}[1]{\textcolor[rgb]{0.02,0.16,0.49}{{#1}}}
    \newcommand{\RegionMarkerTok}[1]{{#1}}
    \newcommand{\ErrorTok}[1]{\textcolor[rgb]{1.00,0.00,0.00}{\textbf{{#1}}}}
    \newcommand{\NormalTok}[1]{{#1}}
    
    % Additional commands for more recent versions of Pandoc
    \newcommand{\ConstantTok}[1]{\textcolor[rgb]{0.53,0.00,0.00}{{#1}}}
    \newcommand{\SpecialCharTok}[1]{\textcolor[rgb]{0.25,0.44,0.63}{{#1}}}
    \newcommand{\VerbatimStringTok}[1]{\textcolor[rgb]{0.25,0.44,0.63}{{#1}}}
    \newcommand{\SpecialStringTok}[1]{\textcolor[rgb]{0.73,0.40,0.53}{{#1}}}
    \newcommand{\ImportTok}[1]{{#1}}
    \newcommand{\DocumentationTok}[1]{\textcolor[rgb]{0.73,0.13,0.13}{\textit{{#1}}}}
    \newcommand{\AnnotationTok}[1]{\textcolor[rgb]{0.38,0.63,0.69}{\textbf{\textit{{#1}}}}}
    \newcommand{\CommentVarTok}[1]{\textcolor[rgb]{0.38,0.63,0.69}{\textbf{\textit{{#1}}}}}
    \newcommand{\VariableTok}[1]{\textcolor[rgb]{0.10,0.09,0.49}{{#1}}}
    \newcommand{\ControlFlowTok}[1]{\textcolor[rgb]{0.00,0.44,0.13}{\textbf{{#1}}}}
    \newcommand{\OperatorTok}[1]{\textcolor[rgb]{0.40,0.40,0.40}{{#1}}}
    \newcommand{\BuiltInTok}[1]{{#1}}
    \newcommand{\ExtensionTok}[1]{{#1}}
    \newcommand{\PreprocessorTok}[1]{\textcolor[rgb]{0.74,0.48,0.00}{{#1}}}
    \newcommand{\AttributeTok}[1]{\textcolor[rgb]{0.49,0.56,0.16}{{#1}}}
    \newcommand{\InformationTok}[1]{\textcolor[rgb]{0.38,0.63,0.69}{\textbf{\textit{{#1}}}}}
    \newcommand{\WarningTok}[1]{\textcolor[rgb]{0.38,0.63,0.69}{\textbf{\textit{{#1}}}}}
    
    
    % Define a nice break command that doesn't care if a line doesn't already
    % exist.
    \def\br{\hspace*{\fill} \\* }
    % Math Jax compatability definitions
    \def\gt{>}
    \def\lt{<}
    % Document parameters
    \title{Assignment 2}
    
    
    
% Pygments definitions
    
    \makeatletter
    \newcommand*\@iflatexlater{\@ifl@t@r\fmtversion}
    \@iflatexlater{2016/03/01}{
	    \newcommand{\wordboundary}{4095}}{
	    \newcommand{\wordboundary}{255}}
    \makeatother

    \newif\ifcode
    \codefalse
    \definecolor{Grey}{rgb}{0.40,0.40,0.40}
    %If using XeLaTeX, use magic to not highlight . operators with purple.
    \ifdefined\XeTeXcharclass
    \XeTeXinterchartokenstate = 1
    \newXeTeXintercharclass \mycharclassGrey
    \XeTeXcharclass `. \mycharclassGrey
    \XeTeXinterchartoks 0 \mycharclassGrey   = {\bgroup\ifcode\color{Grey}\else\fi}

    \XeTeXinterchartoks \wordboundary \mycharclassGrey = {\bgroup\ifcode\color{Grey}\else\fi}

    \XeTeXinterchartoks \mycharclassGrey 0   = {\egroup}
    \XeTeXinterchartoks \mycharclassGrey \wordboundary = {\egroup}
    \fi %end magical operator highlighting
    %End Reconfigured Pygments
    
   
    % Exact colors from NB
    \definecolor{incolor}{HTML}{303F9F}
    \definecolor{outcolor}{HTML}{D84315}
    \definecolor{cellborder}{HTML}{CFCFCF}
    \definecolor{cellbackground}{HTML}{F7F7F7}

    % needed definitions
    \newif\ifleftmargins
    \newlength{\promptlength}

    % cell style settings
        \leftmarginsfalse

    
    % prompt
    \newcommand{\prompt}[3]{
        \needspace{1.1cm}
        \settowidth{\promptlength}{ #1 [#3] }
        \ifleftmargins\hspace{-\promptlength}\hspace{-5pt}\fi
        {\color{#2}#1 [#3]:}
        \ifleftmargins\vspace{-2.7ex}\fi
    }
    
    
    % environments
    \newenvironment{OutVerbatim}{\VerbatimEnvironment%
        \begin{tcolorbox}[breakable, boxrule=.5pt, size=fbox, pad at break*=1mm, opacityfill=0]
            \begin{Verbatim}
            }{
            \end{Verbatim}
        \end{tcolorbox}
    }
    
    %Updated MathJax Compatibility (if future behaviour of the notebook changes this may be removed)
    \renewcommand{\TeX}{\ifmmode \textrm{\Oldtex} \else \textbackslash TeX \fi}
    \renewcommand{\LaTeX}{\ifmmode \Oldlatex \else \textbackslash LaTeX \fi}
    
    % Header Adjustments
    \renewcommand{\paragraph}{\textbf}
    \renewcommand{\subparagraph}[1]{\textit{\textbf{#1}}}

    
    % Prevent overflowing lines due to hard-to-break entities
    \sloppy 
    % Setup hyperref package
    \hypersetup{
      breaklinks=true,  % so long urls are correctly broken across lines
      colorlinks=true,
      urlcolor=urlcolor,
      linkcolor=linkcolor,
      citecolor=citecolor,
      }
    % Slightly bigger margins than the latex defaults
    \geometry{verbose,tmargin=.5in,bmargin=.7in,lmargin=.5in,rmargin=.5in}
    

    \begin{document}
    
    
    
    
    

    
    \hypertarget{assignment-2}{%
\section{Assignment 2}\label{assignment-2}}

    
\prompt{In}{incolor}{201}
\codetrue
\begin{tcolorbox}[breakable, size=fbox, boxrule=1pt, pad at break*=1mm, colback=cellbackground, colframe=cellborder]
\begin{minted}[breaklines=True]{ipython3}
import numpy as np
import matplotlib.pyplot as plt
from numpy import pi, sin
from textwrap import wrap
\end{minted}
\end{tcolorbox}
\codefalse

    \hypertarget{a-potential-flow}{%
\subsection{A potential flow}\label{a-potential-flow}}

    
\prompt{In}{incolor}{221}
\codetrue
\begin{tcolorbox}[breakable, size=fbox, boxrule=1pt, pad at break*=1mm, colback=cellbackground, colframe=cellborder]
\begin{minted}[breaklines=True]{ipython3}
alpha = pi/3
theta = np.linspace(10**-100, alpha-10**-100, 1000)
n = 1

fig = plt.figure(dpi = 300)
ax = fig.add_subplot(111, projection='polar')

for C in range(100):
    c = C*0.1 # To space out the lines
    s = sin(n*pi*theta/alpha)
    r = (C*c*n/s)**(alpha/(n*pi))
    ax.plot(theta, r)

Title = wrap(r"Stream lines for a non-compessable fluid"
             r"in a wedge of angle $\frac{\pi}{3}$ rad")

plt.title("\n".join(Title))
ax.set_thetamin(0)
ax.set_thetamax(alpha*180/pi)
ax.set_ylim(0, 10)
\end{minted}
\end{tcolorbox}
\codefalse
 
            
\prompt{Out}{outcolor}{221}
    
    $$\left ( 0, \quad 10\right )$$

    

    \begin{center}
    \adjustimage{max size={0.9\linewidth}{0.9\paperheight}}{output_3_1.png}
    \end{center}
    { \hspace*{\fill} \\}
    
    \hypertarget{contour-intergals-in-electromagnetism}{%
\subsection{Contour intergals in
electromagnetism}\label{contour-intergals-in-electromagnetism}}

    
\prompt{In}{incolor}{70}
\codetrue
\begin{tcolorbox}[breakable, size=fbox, boxrule=1pt, pad at break*=1mm, colback=cellbackground, colframe=cellborder]
\begin{minted}[breaklines=True]{ipython3}
import sympy as sym 
import sympy.vector
from sympy import Derivative, symbols, simplify, sqrt, I, factorial, pi
from sympy.physics.mechanics import *
init_vprinting()
\end{minted}
\end{tcolorbox}
\codefalse

    \hypertarget{integral-that-needs-evaluation}{%
\subsubsection{Integral that needs
evaluation}\label{integral-that-needs-evaluation}}

\(\int_0^P \frac{\cos^2u}{[1+C\sin u]^5} du\) =
\(\frac{2^3}{C^5} \int_0^{2\pi} \frac{(z^2+1)^2 z^2}{(z-z_-)^5(z-z_+)^5} dz\)

    
\prompt{In}{incolor}{42}
\codetrue
\begin{tcolorbox}[breakable, size=fbox, boxrule=1pt, pad at break*=1mm, colback=cellbackground, colframe=cellborder]
\begin{minted}[breaklines=True]{ipython3}
beta, theta, C, z = symbols('beta theta C z')
\end{minted}
\end{tcolorbox}
\codefalse

    
\prompt{In}{incolor}{43}
\codetrue
\begin{tcolorbox}[breakable, size=fbox, boxrule=1pt, pad at break*=1mm, colback=cellbackground, colframe=cellborder]
\begin{minted}[breaklines=True]{ipython3}
# C = beta * cos(theta)
z_p = (-I + I*sqrt(1-C**2))/C
z_m = (-I - I*sqrt(1-C**2))/C
\end{minted}
\end{tcolorbox}
\codefalse

    \hypertarget{phiz-fracz2z212z---z_-5}{%
\paragraph{\texorpdfstring{\(\Phi(z) = \frac{z^2(z^2+1)^2}{(z - z_-)^5}\)}{\textbackslash{}Phi(z) = \textbackslash{}frac\{z\^{}2(z\^{}2+1)\^{}2\}\{(z - z\_-)\^{}5\}}}\label{phiz-fracz2z212z---z_-5}}

    
\prompt{In}{incolor}{79}
\codetrue
\begin{tcolorbox}[breakable, size=fbox, boxrule=1pt, pad at break*=1mm, colback=cellbackground, colframe=cellborder]
\begin{minted}[breaklines=True]{ipython3}
Coeff = 2**3 / C**5
IntTop = (z**2 + 1)**2 * z**2
IntDwn = (z-z_m)**5 * (z-z_p)**5
Int = IntTop/IntDwn
Phi = IntTop/(z-z_m)**5
phi = Phi.simplify()
phi
\end{minted}
\end{tcolorbox}
\codefalse
 
            
\prompt{Out}{outcolor}{79}
    
    $$\frac{z^{2} \left(z^{2} + 1\right)^{2}}{\left(z - \frac{- i \sqrt{- C^{2} + 1} - i}{C}\right)^{5}}$$

    

    \hypertarget{fracd4dz4-phiz}{%
\paragraph{\texorpdfstring{\(\frac{d^4}{dz^4} \Phi(z)\)}{\textbackslash{}frac\{d\^{}4\}\{dz\^{}4\} \textbackslash{}Phi(z)}}\label{fracd4dz4-phiz}}

    
\prompt{In}{incolor}{74}
\codetrue
\begin{tcolorbox}[breakable, size=fbox, boxrule=1pt, pad at break*=1mm, colback=cellbackground, colframe=cellborder]
\begin{minted}[breaklines=True]{ipython3}
D = Derivative(phi, z, 4).doit().simplify()
D
\end{minted}
\end{tcolorbox}
\codefalse
 
            
\prompt{Out}{outcolor}{74}
    
    $$\frac{24 C^{5} \left(70 C^{4} z^{2} \left(z^{2} + 1\right)^{2} - 70 C^{3} z \left(z^{2} + 1\right) \left(3 z^{2} + 1\right) \left(C z + i \sqrt{- C^{2} + 1} + i\right) + 15 C^{2} \left(4 z^{4} + 10 z^{2} \left(z^{2} + 1\right) + \left(z^{2} + 1\right)^{2}\right) \left(C z + i \sqrt{- C^{2} + 1} + i\right)^{2} - 20 C z \left(5 z^{2} + 2\right) \left(C z + i \sqrt{- C^{2} + 1} + i\right)^{3} + \left(15 z^{2} + 2\right) \left(C z + i \sqrt{- C^{2} + 1} + i\right)^{4}\right)}{\left(C z + i \sqrt{- C^{2} + 1} + i\right)^{9}}$$

    

    \hypertarget{evaluating-fracd4dz4-phiz-at-z-z_}{%
\paragraph{\texorpdfstring{Evaluating \(\frac{d^4}{dz^4} \Phi(z)\) at
\(z = z_+\)}{Evaluating \textbackslash{}frac\{d\^{}4\}\{dz\^{}4\} \textbackslash{}Phi(z) at z = z\_+}}\label{evaluating-fracd4dz4-phiz-at-z-z_}}

    
\prompt{In}{incolor}{78}
\codetrue
\begin{tcolorbox}[breakable, size=fbox, boxrule=1pt, pad at break*=1mm, colback=cellbackground, colframe=cellborder]
\begin{minted}[breaklines=True]{ipython3}
DVal = D.subs(z, z_p).simplify()
DVal
\end{minted}
\end{tcolorbox}
\codefalse
 
            
\prompt{Out}{outcolor}{78}
    
    $$\frac{3 i C^{5} \left(C^{2} + 4\right)}{8 \sqrt{- C^{2} + 1} \left(C^{6} - 3 C^{4} + 3 C^{2} - 1\right)}$$

    

    \hypertarget{heres-the-residue}{%
\paragraph{Here's the residue}\label{heres-the-residue}}

\$ \frac{1}{4!} \frac{d^4}{dz^4} \Phi(z\_+)\$

    
\prompt{In}{incolor}{77}
\codetrue
\begin{tcolorbox}[breakable, size=fbox, boxrule=1pt, pad at break*=1mm, colback=cellbackground, colframe=cellborder]
\begin{minted}[breaklines=True]{ipython3}
Res = DVal / factorial(4)
Res
\end{minted}
\end{tcolorbox}
\codefalse
 
            
\prompt{Out}{outcolor}{77}
    
    $$\frac{i C^{5} \left(C^{2} + 4\right)}{64 \sqrt{- C^{2} + 1} \left(C^{6} - 3 C^{4} + 3 C^{2} - 1\right)}$$

    

    \hypertarget{value-of-integral}{%
\paragraph{Value of integral}\label{value-of-integral}}

    
\prompt{In}{incolor}{65}
\codetrue
\begin{tcolorbox}[breakable, size=fbox, boxrule=1pt, pad at break*=1mm, colback=cellbackground, colframe=cellborder]
\begin{minted}[breaklines=True]{ipython3}
Integral = 2*pi*I*Res * Coeff
Integral.simplify()
\end{minted}
\end{tcolorbox}
\codefalse
 
            
\prompt{Out}{outcolor}{65}
    
    $$- \frac{\pi \left(C^{2} + 4\right)}{4 \sqrt{- C^{2} + 1} \left(C^{6} - 3 C^{4} + 3 C^{2} - 1\right)}$$

    


    % Add a bibliography block to the postdoc
    
    
    
    \end{document}
